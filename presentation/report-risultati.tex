\documentclass[11pt]{article}
\usepackage[utf8]{inputenc}
\usepackage[italian]{babel}
\usepackage{amsmath}
\usepackage{amsfonts}
\usepackage{amssymb}
\usepackage{wasysym}
\usepackage{indentfirst}
\usepackage[shortlabels]{enumitem}
\usepackage{float}

\author{Barbaro, Bifulco, Sansonetti}
\title{Progetto CLIPS}
\date{A.A. 2022/2023}
% \institute[]{Università degli studi di Torino\\Intelligenza Artificiale e Laboratorio}  

\begin{document}

\maketitle

\section*{Notazione utilizzata}
\subsection*{Stategie}
Elenco delle strategie testate su ogni board per ogni livello di osservabilità considerato.

\begin{enumerate}[a.]
    \item $\langle 1, 1, 1, 1, 1 \rangle$
    \item $\langle 3, 2 \rangle$
    \item $\langle 5 \rangle$
\end{enumerate}

\subsection*{Osservabilità testate}
Elenco dei casi di osservabilità testati, nei capitolo corrispondenti alle singole board verranno indicate le celle note.

Nei risultati ci si riferirà ai livelli di asservabilità con i numeri sottostanti.

\begin{enumerate}
    \item Osservabilità nulla
    \item 5 acque 
    \item Noti 3 pezzi scorrelati tra loro
    \item Noti solo i sottomarini
    \item Noto un pezzo \emph{middle} per ogni nave
    \item Noto un estremo per ogni nave
\end{enumerate}

\section{Caso 1}
\subsection{Board}
\begin{table}[H]
    \begin{tabular}{|l|l|l|l|l|l|l|l|l|l||l|}
    \hline
     $\Circle$ &  &  &  &  &  &  & $\subset$ & $\square$ & $\supset$ & 4 \\ \hline
     &  &  &  &  & $\cap$ &  &  &  &  & 1 \\ \hline
     &  &  &  &  & $\cup$ &  &  & $\Circle$ &  & 2 \\ \hline
     &  &  &  &  &  &  &  &  &  & 0 \\ \hline
     &  &  &  &  & $\cap$ &  &  &  &  & 1 \\ \hline
     &  & $\subset$ & $\supset$ &  & $\square$ &  & $\subset$ & $\square$ & $\supset$ & 6 \\ \hline
     &  &  &  &  & $\square$ &  &  &  &  & 1 \\ \hline
     &  &  &  &  & $\cup$ &  &  &  &  & 1 \\ \hline
     $\cap$ &  &  &  &  &  &  &  & $\Circle$ &  & 2 \\ \hline
     $\cup$ &  &  &  &  & $\Circle$ &  &  &  &  & 2 \\ \hline \hline
     3 & 0 & 1 & 1 & 0 & 7 & 0 & 2 & 4 & 2 & \\ \hline
    \end{tabular}
\end{table}

\subsection{Celle note}
\begin{enumerate}
    \item Osservabilità nulla
    \item 5 acque 
    \item Noti 3 pezzi scorrelati tra loro
    \item Noti solo i sottomarini
    \item Noto un pezzo \emph{middle} per ogni nave
    \item Noto un estremo per ogni nave
\end{enumerate}

\subsection{Risultati}
\begin{table}[H]
    \begin{tabular}{|c|c|c|c|}
    \hline
    x & a & b & c \\ \hline \hline
    1 &  &  &  \\ \hline
    2 &  &  &  \\ \hline
    3 &  &  &  \\ \hline
    4 &  &  &  \\ \hline
    5 &  &  &  \\ \hline
    6 &  &  &  \\ \hline
    \end{tabular}
\end{table}

\section{Caso 2}
\subsection{Board}
\begin{table}[H]
    \begin{tabular}{|l|l|l|l|l|l|l|l|l|l||l|}
    \hline
     $\cap$ &  & $\cap$ &  &  &  &  & $\cap$ &  &  & 3 \\ \hline
     $\square$ &  & $\cup$ &  &  &  &  & $\cup$ &  &  & 3 \\ \hline
     $\square$ &  &  &  &  &  &  &  &  &  & 1 \\ \hline
     $\cup$ &  & $\Circle$ &  &  &  &  &  &  &  & 2 \\ \hline
     &  &  &  &  &  &  &  &  &  & 0 \\ \hline
     $\cap$ &  & $\cap$ &  &  &  &  & $\cap$ &  &  & 3 \\ \hline
     $\square$ &  & $\square$ &  &  &  &  & $\cup$ &  &  & 3 \\ \hline
     $\cup$ &  & $\cup$ &  &  &  &  &  &  &  & 2 \\ \hline
     &  &  &  &  &  &  &  &  &  & 0 \\ \hline
     $\Circle$ &  & $\Circle$ &  &  &  &  & $\Circle$ &  &  & 3 \\ \hline \hline
     8 & 0 & 7 & 0 & 0 & 0 & 0 & 5 & 0 & 0 & \\ \hline
    \end{tabular}
\end{table}

\subsection{Celle note}
\begin{enumerate}
    \item Osservabilità nulla
    \item 5 acque 
    \item Noti 3 pezzi scorrelati tra loro
    \item Noti solo i sottomarini
    \item Noto un pezzo \emph{middle} per ogni nave
    \item Noto un estremo per ogni nave
\end{enumerate}

\subsection{Risultati}
\begin{table}[H]
    \begin{tabular}{|c|c|c|c|}
    \hline
    x & a & b & c \\ \hline \hline
    1 &  &  &  \\ \hline
    2 &  &  &  \\ \hline
    3 &  &  &  \\ \hline
    4 &  &  &  \\ \hline
    5 &  &  &  \\ \hline
    6 &  &  &  \\ \hline
    \end{tabular}
\end{table}

\section{Caso 3}
\subsection{Board}
\begin{table}[H]
    \begin{tabular}{|l|l|l|l|l|l|l|l|l|l||l|}
    \hline
     $\Circle$ &  &  &  &  &  &  &  & $\Circle$ &  & 2 \\ \hline
     &  &  & $\subset$ & $\supset$ &  &  &  &  &  & 2 \\ \hline
     &  &  &  &  &  &  & $\Circle$ &  &  & 1 \\ \hline
     & $\subset$ & $\supset$ &  &  &  &  &  &  &  & 2 \\ \hline
     &  &  &  &  & $\subset$ & $\supset$ &  &  &  & 2 \\ \hline
     &  &  & $\Circle$ &  &  &  &  &  &  & 1 \\ \hline
     &  &  &  &  &  &  & $\subset$ & $\square$ & $\supset$ & 3 \\ \hline
     $\cap$ &  &  &  &  &  &  &  &  &  & 1 \\ \hline
     $\square$ &  &  & $\subset$ & $\square$ & $\square$ & $\supset$ &  &  &  & 5 \\ \hline
     $\cup$ &  &  &  &  &  &  &  &  &  & 1 \\ \hline \hline
     4 & 1 & 1 & 3 & 2 & 2 & 2 & 2 & 2 & 1 & \\ \hline
    \end{tabular}
\end{table}

\subsection{Celle note}
\begin{enumerate}
    \item Osservabilità nulla
    \item 5 acque 
    \item Noti 3 pezzi scorrelati tra loro
    \item Noti solo i sottomarini
    \item Noto un pezzo \emph{middle} per ogni nave
    \item Noto un estremo per ogni nave
\end{enumerate}

\subsection{Risultati}
\begin{table}[H]
    \begin{tabular}{|c|c|c|c|}
    \hline
    x & a & b & c \\ \hline \hline
    1 &  &  &  \\ \hline
    2 &  &  &  \\ \hline
    3 &  &  &  \\ \hline
    4 &  &  &  \\ \hline
    5 &  &  &  \\ \hline
    6 &  &  &  \\ \hline
    \end{tabular}
\end{table}


\end{document}