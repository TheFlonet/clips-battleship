\documentclass[11pt]{beamer}
\usetheme{Madrid}
\usepackage[utf8]{inputenc}
\usepackage[italian]{babel}
\usepackage{amsmath}
\usepackage{amsfonts}
\usepackage{amssymb}
\usepackage{graphicx}
\usepackage{pgfplots}
\usepackage{wasysym}
\pgfplotsset{width=9.5cm,compat=1.18}

\author{Barbaro, Bifulco, Sansonetti}
\title{Progetto CLIPS}
\setbeamertemplate{navigation symbols}{}
\date{A.A. 2022/2023}
\institute[]{Università degli studi di Torino\\Intelligenza Artificiale e Laboratorio}  

\begin{document}

\begin{frame}
\titlepage
\end{frame}

% \begin{frame}{Indice}
% \tableofcontents 
% \end{frame}

\begin{frame}
    \frametitle{Problema affrontato}
    Sviluppare un sistema esperto in grado di giocare ad una versione semplificata della battaglia navale.
\end{frame}

\begin{frame}
    \frametitle{Rappresentazione della conoscenza}
    Esempio di board utilizzata:
    \begin{table}
        \begin{tabular}{|l|l|l|l|l|l|l|l|l|l||l|}
        \hline
         &  &  &  &  &  &  &  &  &  & \\ \hline
         &  &  &  &  &  & $\subset$ & $\supset$ &  &  & \\ \hline
         &  &  &  &  &  &  &  &  &  & \\ \hline
         &  &  &  &  &  &  &  &  &  & \\ \hline
         &  & $\Circle$ &  &  &  &  &  &  &  & \\ \hline
         &  &  &  &  &  & $\cap$ &  &  &  & \\ \hline
         &  &  &  &  &  & $\cup$ &  &  &  & \\ \hline
         &  &  &  &  &  &  &  &  &  & \\ \hline
         &  &  &  &  &  &  &  &  &  & \\ \hline
         &  &  &  &  & $\square$ &  &  &  &  & \\ \hline \hline
         &  &  &  &  &  &  &  &  &  & \\ \hline
        \end{tabular}
    \end{table}
\end{frame}

\begin{frame}
    \frametitle{Ragionamenti sui dati in input}
    \begin{itemize}
        \item Inserimento della conoscenza fornita dall'ambiente
        \item Posizionamento dell'acqua intorno ai pezzi di nave
        \item Posizionamento dell'acqua su colonne/righe vuote
        \item Posizionamento dell'acqua su righe/colonne già note
        \item Assert di pezzi di nave certi
    \end{itemize}
\end{frame}

\begin{frame}
    \frametitle{Strategia di risoluzione}
    All'agente viene fornita una strategia di risoluzione che permette di alternare fasi di \emph{guess} a fasi in cui usare le \emph{fire}.

    Questo meccanismo permette di dosare quanto si vuole rendere l'agente aggressivo nella risoluzione:
    \begin{itemize}
        \item $\langle 1, 1, 1, 1, 1 \rangle$ produrrà un comportamento più conservativo
        \item $\langle 5 \rangle$ produrrà una strategia più \emph{greedy}
    \end{itemize}
\end{frame}

\begin{frame}
    \frametitle{Inferenze 1/3}
    Durante l'esecuzione il sistema esperto è in grado di piazzare delle \emph{guess} anche se ancora non sa con certezza a quale pezzo di nave corrisponderà.

    Per questo motivo ad ogni round vengono piazzate guess certe su:
    \begin{itemize}
        \item Pezzi di nave noti
        \item Celle immediatamente a lato di un estremo
        \item Ai lati di un pezzo \emph{middle} se c'è modo di dedurre l'orientamento
    \end{itemize}
\end{frame}

\begin{frame}
    \frametitle{Inferenze 2/3 - Lancio delle fire}
    Il sistema esperto richiede all'ambiente maggiore informazione in modo da poter inferire il maggior numero di informazioni:
    \begin{enumerate}
        \item Cella blank a lato di un pezzo \emph{middle} di cui non si conosce l'orientamento
        \item 2 celle oltre un estremo di nave 
        \item 2 celle oltre un pezzo middle di cui si conosce l'orientamento
        \item A lato di sequenze \emph{estremo-middle}
    \end{enumerate}

    Se non è possibile eseguire una delle regole precedenti viene eseguita fire su una cella \emph{blank}
\end{frame}

\begin{frame}
    \frametitle{Inferenze 3/3}
    Se il sistema esperto ha esaurito le \emph{fire} e non è riuscito a piazzare tutte le \emph{guess} posiziona sulle celle che più probabilmente conterranno pezzi di nave.
    \begin{enumerate}
        \item 2 celle oltre un estremo di nave 
        \item Intorno ai pezzi middle
        \item Nelle celle ancora \emph{blank}
    \end{enumerate}
\end{frame}

\begin{frame}
    \frametitle{Risultati 1/2}
    Il sistema esperto è stato testato con la board iniziale sui seguenti casi:
    \begin{enumerate}
        \item Osservabilità nulla
        \item 5 acque 
        \item Noti 3 pezzi scorrelati tra loro
        \item Noti solo i sottomarini
        \item Noto un pezzo \emph{middle} per ogni nave
        \item Noto un estremo per ogni nave
    \end{enumerate}
\end{frame}

\begin{frame}
    \frametitle{Risultati 2/2 - Punteggi}
    \begin{tikzpicture}
        \begin{axis}[
            xlabel={Casi},
            ylabel={Punteggi},
            xmin=1, xmax=5,
            ymin=-500, ymax=500,
            xtick={1,2,3,4,5},
            % ytick={0,20,40,60,80,100,120},
            ymajorgrids=true,
            grid style=dashed,
        ]
        
            \addplot[color=blue, mark=square]
            coordinates {(1, 50)(2, 50)(3, 50)(4, 50)(5, 50)};
        \end{axis}
    \end{tikzpicture}
\end{frame}

\end{document}